\ifx\isEmbedded\undefined
\documentclass[12pt]{article}
	
% FONT RELATED
%\usepackage{times} %Move to times font
\usepackage[labelfont=bf,textfont=it]{caption}


% LINKS, PAGE OF CONTENT, REF AND CROSS-REF, HEADERS/FOOTERS
\usepackage{hyperref}
\hypersetup{
    colorlinks,
    citecolor=black,
    filecolor=black,
    linkcolor=black,
    urlcolor=black
}
%\usepackage[breaklinks=true]{hyperref}
\usepackage{fancyhdr}
\usepackage{acronym}

% FIGURES, GRAPHICS, TABLES
\usepackage{graphicx}
\usepackage{parskip}
\usepackage{subfigure}

% COLOURS, TEXT AND FORMATTING
\usepackage{array}
\usepackage{color}
\usepackage{setspace}
\usepackage{longtable}
\usepackage{multirow}

% ADVANCED MATHS, PSEUDO-CODE
\usepackage{amsmath}
\usepackage{alltt}
\usepackage{amsfonts}
\usepackage{listings}
\usepackage{amsmath}

% BIBLIOGRAPHY
\usepackage[authoryear]{natbib}
\bibpunct{(}{)}{;}{a}{}{,}

% USE IN DISSER:

\setlength\oddsidemargin{1.5cm}
\setlength\evensidemargin{5cm}

\setlength\textheight{9.0in}
\setlength\textwidth{5.1in}

% indent at each new paragrapg
\setlength\parindent{0.5cm}

\setlength\topmargin{-0.2in}
\renewcommand{\baselinestretch}{1.3}

%REPORT

%\setlength\oddsidemargin{1cm}
%\setlength\evensidemargin{0.3in}
%%\setlength\headsep{2.5in}
%
%\setlength\textheight{9.0in}
%\setlength\textwidth{5.5in}
%
%% indent at each new paragrapg
%\setlength\parindent{0.5cm}
%
%%\setlength{\parskip}{10.5ex}
%
%\setlength\topmargin{-0.2in}

%\newcommand{\HRule}{\rule{\linewidth}{0.5mm}}
\newcommand{\HRule}{\rule{\linewidth}{0.0mm}}

%% macros
\newcommand{\RR}{\mathbb{R}} 
\newcommand{\pt}[1]{\mathbf{#1}} 

% Color definitions (RGB model)
\definecolor{ms-comment}{rgb}{0.1, 0.4, 0.1}
\definecolor{ms-question}{rgb}{0.8, 0.2, 0.2}
\definecolor{ms-new}{rgb}{0.2, 0.4, 0.8}

\newcommand\red[1]{{\color{red}#1}}
\newcommand\blue[1]{{\color{blue}#1}}
\newcommand\comment[1]{{\iffalse #1 \fi}}

\setcounter{secnumdepth}{3}
\setcounter{tocdepth}{3} 

\graphicspath{{../img/}}
\begin{document}
\tableofcontents
\pagebreak

\section*{List of Acronyms}
\addcontentsline{toc}{section}{List of Acronyms}

\begin{acronym}[NURBS ]
\acro{AABB}{Axis Aligned Bounding Box}
\acro{ADF}{Adaptively sampled Distance Field}
\acro{BRep}{Boundary Representation}
\acro{BReps}{Boundary Representations}
\acro{BVH}{Bounding Volume Hierarchy}
\acro{CAD}{Computer-Aided Design}
\acro{CSG}{Constructive Solid Geometry}
\acro{FRep}{Function Representation}
\acro{GPU}{Graphics Processing Unit}
\acro{MVC}{Mean Value Coordinates}
\acro{NURBS}{Non-Uniform Rational B-Spline}
\acro{OBB}{Oriented Bounding Box}
\acro{PCA}{Principal Component Analysis}
\acro{RBF}{Radial Basis Function}
\acro{SAH}{Surface Area Heuristic}
\acro{SDF}{Signed Distance Field}
\acro{STTI}{Space Time Transfinite Interpolation}
\acro{SARDF}{Signed Approximate Real Distance Functions}
\acro{TI}{Transfinite Interpolation}
\end{acronym}

\pagebreak
\fi

\section{Physics} \label{sec_physics}
In order to simulate granular materials, it is necessary to define what they are and understand how they physically behave. For this reason, this section will look at how granular materials are described in the physics and engineering literature rather than the computer graphics literature. However, the scope of this report clearly does not allow a thorough discussion of the physics of granular matter. It will give an introduction, geared towards the techniques in the computer graphics community, which are the subject of the next sections. It is interesting to add that the physics of granular matter is still a subject of debate DEGENNES.\\

As a first step towards understanding, we need to define what granular material is. Referring back to the physics literature, de Gennes (REF) defines a granular material as a material made up out of a large collection of grains or particles, where the size $d$ (for diameter) of the individual grains are larger than one micron. At this scale, the Brownian motion and thermal agitation can be ignored. deGennesREF. This means that entropy can be ignored and the behaviour is determined by dynamical effects. JAEGERREF\\

Hence, to understand the behaviour, it necessary to investigate the dynamics of the material. The first important property is the friction, both static and dynamic, between the different grains. Coulomb's law is a widely accepted and well-known way of modelling (dry) friction between two surfaces in contact:
\begin{equation} \label{phys_friction}
F_f \leq \mu F_n
\end{equation} 
where $F_f$ is the friction force, tangential to the contact surface, $F_n$ the force perpendicular to the contact surface, and $\mu$ the coefficient of friction. Notice the relation, $\leq$, indicating that the friction force will be as big as necessary to avoid sliding in the case of static friction, i.e. the friction force will exactly oppose sliding and keep the objects in the same position relative to each other. In the case of kinetic sliding, the $\leq$ symbol disappears, making it an equality. Hence, when the objects are sliding relative to each other, kinetic friction results in a constant force opposing the sliding motion. Typically, the friction coefficient $\mu$ is different for static $\mu_s$ and kinetic friction $\mu_k$.\\

Another property of granular material is the inelastic collision between the particles JAEGERREF. This, combined with the friction, makes that granular material are dissipative systems, where a lot of kinetic energy is converted in heat rather than bounces JEAGERREFHowever. It is interesting to note that the some grains might be subject to an infinite amount of collisions in a finite amount of time, a phenomenon called inelastic collapse McNamara and Young REF.\\

All of this results in some quite particular behaviour. For instance, a well-known result of these physics is that the pressure in a granular material does not linearly increase with height as it does in fluid, but stay constant after a certain depth. REFS This phenomenon is what makes hourglasses work. REFHOURGLASS. The static friction also makes the sand pile up, without the piles collapsing, along as the slope of the pile is less steep than the angle of repose. JAEGERREF. If the slope is steeper, an avalanche will occur, where only the top layer of the granular material flows. ARANSONREF\\

These physics, and the particular behaviour of granular matter, lead physicists to describe them as a material with both solid-like and fluid-like behaviour. DEGENNESJAEGERARANSONREF. When approaching the material as a fluid, a continuum model is necessary. This continuum model is usually formulated using a model based on a variation of the Navier-Stokes equations, adapted to model the granular behavior, and complemented with a elasto-plastic formulation of a yield criterion, for example the Mohr-Coulomb yield criterion. DEGENNESREF The Mohr-Coulomb yield criterion is defined as (notation from REFZAB):
\begin{equation} \label{mohr_phys}
\sqrt{3} \frac{|\sigma - \sigma_m \delta |_F}{\sqrt{2}} < sin \phi \sigma_m,
\end{equation}
where $\sigma$ is the stress tensor, $\sigma_m$ is the mean stress and $\phi$ is the friction angle (angle of repose). The norm taken in the left side is the Frobenius norm, denoted by the subscript. This equation means that the shear stress ($\sigma -\sigma_m \delta$) in the left side is ignored because of friction when the pressure $\sigma_m$ pushing the material together is high enough, and hence governs the difference between the solid-like and fluid-like behaviour of the granular matter (that is, the material is assumed to start flowing when it 'yields') DEGENNESREF. This law can be seen as a generalization of the classical Coulomb friction to stress tensors, where the friction coefficient is replaced by the sine of the friction angle. By using this rule to determine when the medium starts to flow, ZABREF incorporate static friction into a continuum model.\\

While discussing the necessary changes to the Navier-Stokes equations to model granular material is interesting in its own rights, this would be out of scope for this report, as the computer graphics techniques usually bring it down to two particular differences from regular fluid simulations REFTOTECH: one needs to account for the friction and a specific form of (in)compressibility: dry granular materials are usually simulated with unilateral incompressibility, while wet granular materials, e.g. snow, are typically compressible. Unilateral incompressibility means that there is no cohesive pressure when the material is diverging REFTECH. The computer graphics techniques based on the continuum approach usually take existing fluid simulators and change these two aspects. The specific formulations for the two differ per technique and will be given while explaining specific techniques in the next section.\\

\ifx\isEmbedded\undefined
% References
\addcontentsline{toc}{section}{References}
\bibliographystyle{../ref/harvardnat}
\bibliography{../ref/master}
\pagebreak
\end{document}
\fi