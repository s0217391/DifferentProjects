
\ifx\isEmbedded\undefined
\documentclass[12pt]{article}
	
% FONT RELATED
%\usepackage{times} %Move to times font
\usepackage[labelfont=bf,textfont=it]{caption}


% LINKS, PAGE OF CONTENT, REF AND CROSS-REF, HEADERS/FOOTERS
\usepackage{hyperref}
\hypersetup{
    colorlinks,
    citecolor=black,
    filecolor=black,
    linkcolor=black,
    urlcolor=black
}
%\usepackage[breaklinks=true]{hyperref}
\usepackage{fancyhdr}
\usepackage{acronym}

% FIGURES, GRAPHICS, TABLES
\usepackage{graphicx}
\usepackage{parskip}
\usepackage{subfigure}

% COLOURS, TEXT AND FORMATTING
\usepackage{array}
\usepackage{color}
\usepackage{setspace}
\usepackage{longtable}
\usepackage{multirow}

% ADVANCED MATHS, PSEUDO-CODE
\usepackage{amsmath}
\usepackage{alltt}
\usepackage{amsfonts}
\usepackage{listings}
\usepackage{amsmath}

% BIBLIOGRAPHY
\usepackage[authoryear]{natbib}
\bibpunct{(}{)}{;}{a}{}{,}

% USE IN DISSER:

\setlength\oddsidemargin{1.5cm}
\setlength\evensidemargin{5cm}

\setlength\textheight{9.0in}
\setlength\textwidth{5.1in}

% indent at each new paragrapg
\setlength\parindent{0.5cm}

\setlength\topmargin{-0.2in}
\renewcommand{\baselinestretch}{1.3}

%REPORT

%\setlength\oddsidemargin{1cm}
%\setlength\evensidemargin{0.3in}
%%\setlength\headsep{2.5in}
%
%\setlength\textheight{9.0in}
%\setlength\textwidth{5.5in}
%
%% indent at each new paragrapg
%\setlength\parindent{0.5cm}
%
%%\setlength{\parskip}{10.5ex}
%
%\setlength\topmargin{-0.2in}

%\newcommand{\HRule}{\rule{\linewidth}{0.5mm}}
\newcommand{\HRule}{\rule{\linewidth}{0.0mm}}

%% macros
\newcommand{\RR}{\mathbb{R}} 
\newcommand{\pt}[1]{\mathbf{#1}} 

% Color definitions (RGB model)
\definecolor{ms-comment}{rgb}{0.1, 0.4, 0.1}
\definecolor{ms-question}{rgb}{0.8, 0.2, 0.2}
\definecolor{ms-new}{rgb}{0.2, 0.4, 0.8}

\newcommand\red[1]{{\color{red}#1}}
\newcommand\blue[1]{{\color{blue}#1}}
\newcommand\comment[1]{{\iffalse #1 \fi}}

\setcounter{secnumdepth}{3}
\setcounter{tocdepth}{3} 

\graphicspath{{../img/}}
\begin{document}
%\maketitle
\fi

\section{Introduction}
Many real-life environments exhibit some form or even multiple forms of granular materials. As they are ubiquitous in reality, virtual environments, such as those in visual effects and animation, should include these in order to be realistic. In this report, techniques for simulating this phenomenon are reviewed.\\

The mechanical behaviour of granular materials is a complex phenomenon that is not yet fully understood and has been a topic of extensive research in physics and engineering. The latter field is interested in the simulation of this behaviour because sand and other granular materials play an important role in many manufacturing processes (REF). While the engineering techniques are very interested in their own right, this report will not touch upon them, as the main focus of our course is visual effects and animation. This is an important difference, as visual effects do not require accuracy but merely visual plausibility, as opposed to the field of engineering.\\

Because of this more relaxed requirements, combined with the complexity of the physics of granular material which aren't fully understood yet JAEGERREF, most visual effect simulation techniques are based on a simplified mathematical model that mimics some or most of the peculiarities of granular media, rather than on the actual physical laws themselves. This report will describe several such models. Although the research into granular media simulation has been going on for a very long time, this report will focus on the more recent techniques.\\

In the next section, granular media are defined, and its physics are briefly reviewed. Section REF describes the several simulation techniques that were investigated. These techniques are then compared in the discussion section, section REF. To broaden the scope of the report, section REF describes some special cases: interaction of granular media with water and a specialized technique for snow, a particular type of granular material. The report then concludes the findings in section REF.\\

\ifx\isEmbedded\undefined
% References
\addcontentsline{toc}{section}{References}
\bibliographystyle{../ref/harvardnat}
\bibliography{../ref/master}
\pagebreak
\end{document}
\fi